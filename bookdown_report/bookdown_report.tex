% !TeX program = pdfLaTeX
\documentclass[12pt]{article}
\usepackage{amsmath}
\usepackage{graphicx,psfrag,epsf}
\usepackage{enumerate}
\usepackage{natbib}
\usepackage{textcomp}
\usepackage[hyphens]{url} % not crucial - just used below for the URL
\usepackage{hyperref}

%\pdfminorversion=4
% NOTE: To produce blinded version, replace "0" with "1" below.
\newcommand{\blind}{0}

% DON'T change margins - should be 1 inch all around.
\addtolength{\oddsidemargin}{-.5in}%
\addtolength{\evensidemargin}{-.5in}%
\addtolength{\textwidth}{1in}%
\addtolength{\textheight}{1.3in}%
\addtolength{\topmargin}{-.8in}%

%% load any required packages here



% tightlist command for lists without linebreak
\providecommand{\tightlist}{%
  \setlength{\itemsep}{0pt}\setlength{\parskip}{0pt}}



\usepackage{booktabs}
\usepackage{longtable}
\usepackage{array}
\usepackage{multirow}
\usepackage{wrapfig}
\usepackage{float}
\usepackage{colortbl}
\usepackage{pdflscape}
\usepackage{tabu}
\usepackage{threeparttable}
\usepackage{threeparttablex}
\usepackage[normalem]{ulem}
\usepackage{makecell}
\usepackage{xcolor}

\begin{document}


\def\spacingset#1{\renewcommand{\baselinestretch}%
{#1}\small\normalsize} \spacingset{1}


%%%%%%%%%%%%%%%%%%%%%%%%%%%%%%%%%%%%%%%%%%%%%%%%%%%%%%%%%%%%%%%%%%%%%%%%%%%%%%

\if0\blind
{
  \title{\bf Three (Groups of) Blind Mice. Familial Clusters of Cataract
Development in Irradiated Mice}

  \author{
        Alyssa Allsop and Amira Burns \\
    Department of Statistics, Colorado State University\\
      }
  \maketitle
} \fi

\if1\blind
{
  \bigskip
  \bigskip
  \bigskip
  \begin{center}
    {\LARGE\bf Three (Groups of) Blind Mice. Familial Clusters of
Cataract Development in Irradiated Mice}
  \end{center}
  \medskip
} \fi

\bigskip
\begin{abstract}
The text of your abstract. 200 or fewer words.
\end{abstract}

\noindent%
{\it Keywords:} Hierarchical modeling, Logistic Regression, Bayesian
analysis
\vfill

\newpage
\spacingset{1.45} % DON'T change the spacing!

\section{Introduction}
\label{sec:intro}

Little is known about the effects of high atomic number and energy (HZE)
radiation, a main component of space radiation to which exposure is
unavoidable beyond Earth's magnetic field. In contrast, extensive
research shows exposure to high doses of gamma radiation leads to acute
radiation sickness. Effects include damage to the blood forming system,
GI system, immune system, increased risk for cancer, cardiovascular
disease, neurodegenerative disease, death, and cataracts.. The serious
health implications from HZE radiation for astronauts who leave Earth's
magnetic field warrant further study. Adverse effects from radiation may
also be attributable to other factors, including genetics; several
different cancers have been observed to cluster in mice families
(Chernyavskiy, et. al., 2017). Accounting for potential family
clustering allows for thorough examination of the primary research
questions. Is there a genetic susceptibility to cataracts caused by
radiation? Accounting for potential genetic susceptibility, is there a
difference in cataract presentation between HZE radiation and gamma
radiation? The full dataset includes 1820 unique mice from 48 unique
families, with equal random assignments by family to each of three
treatment groups. Mice are bred over several generations to create a
genetically heterogeneous sample, the better to represent the diverse
biology of the human population. The treatments are HZE irradiation,
gamma irradiation, and non-irradiated control. The HZE group is
irradiated with either silicon or iron nuclei HZE ions, which are
considered as a single treatment group. The second group is subjected to
137 Cs gamma irradiation. Mice in both irradiated groups are exposed to
radiation at 7-12 weeks of age. The third group is unirradiated control.
All mice are monitored until 800 days of age --effectively a survival
study. Mice are checked weekly for symptoms of cataracts, cataract risk
factors, and other symptoms of radiation exposure such as tumors and
carcinomas. Resources do not allow for weekly measurement of every
mouse; previous measurements are carried forward if a mouse is not
assessed on a particular week. This analysis uses a simplified version
of the data set - a snapshot of the 1169 mice that are alive at 552
days. This cutoff is chosen because it is the median survival time for
the group with the shortest median survival. There are 47 unique
families with n = 396 in the HZE group, n = 277 in the gamma radiation
group, and n = 496 in the unirradiated control group. There are up to
two generations of mice pups from each family in the dataset, but
generation is not distinguished in the simplified data. Family size
ranges from n = 11 to n = 48, with median family size of 24.

\begin{wraptable}{r}{0pt}
\centering
\begin{tabular}{lrrrrr}
  \toprule
\multicolumn{1}{c}{ } & \multicolumn{4}{c}{Cataract Score} & \multicolumn{1}{c}{ } \\
\cmidrule(l{3pt}r{3pt}){2-5}
Treatment & 1 & 2 & 3 & 4 & Total \\ 
  \midrule
Unirradiated & 438 &  45 &  11 &   2 & 496 \\ 
  Gamma & 214 &  53 &   6 &   4 & 277 \\ 
  HZE & 281 & 107 &   6 &   2 & 396 \\ 
   \bottomrule
\end{tabular}
\caption{Counts of score by treatment group} 
\end{wraptable}

The response is Merriam-Focht cataract score (Merriam \& Focht, 1957),
an ordinal categorical variable corresponding to radiation-associated
ocular changes in the eye. A score of 0 is associated with a completely
clear lens, while a score of 5 is associated with a completely occluded
lens. This dataset contains cataract score levels = {[}1, 2, 3, 4{]}. A
score \(\ge\) 2 indicates presence of cataracts, and the small sample
sizes for score \(>\) 2 across all treatment groups raise concerns about
making inference on an ordinal analysis; consequently, the response is
converted to binary \emph{Cataracts} with score = 1 converted to 0, and
score \(\ge\) 2 converted to 1. The main experimental factor is
\emph{Treatment}, a categorical variable with three levels = {[}HZE
radiation, gamma radiation, non-irradiated control{]}. A single random
effect, genetic \emph{Family}, is a categorical factor with 47 levels.
Both Treatment and Family are central to the experimental design and
will serve as the basis for all models considered. Additional covariates
under consideration were Sex, coat color, weight in grams, body
condition score (BCS) age in days, and presence three cancers: myeloid
leukemia, harderian tumors, and PreT lymphoma.

\section{Summary Statistics}
\label{sec:sumstats}

\section{Statistical Methods}
\label{sec:methods}

The experimental design prescribed the inclusion of Treatment and Family
in a final model.

\section{Limitations and Alternatives}
\label{sec:limits}

\section{Results}
\label{sec:results}

\section{Conclusions}
\label{sec:conc}

\section{Author's Statements}
\label{sec:auth}

\section{Appendix}
\label{sec:appendix}

\citet{Campbell02} \citeauthor{Schubert13}
\citetext{\citeyear{Schubert13}; \citealp{Chi81}}

\bibliographystyle{agsm}
\bibliography{bibliography.bib}


\end{document}
